% On découpe ce document complexe en plusieurs sous-fichiers séparés.
% Cela permettra notamment de réarranger les transparents facilement 
% lors de l'élaboration du document.

% La définition de la classe beamer avec tous les styles afférents

\RequirePackage{currfile} 

\documentclass[xcolor=table]{beamer}
\usepackage{animate}
\usepackage{colortbl}

%%% Гиперссылки



\input{preambule/special_beamer.tex}

% Les autres packages utiles  notamment pour le français, les accents ou Python
\input{preambule/autres_packages.tex}
% Les macros et raccourcis personnels
\input{preambule/macros.tex}

% On définit le titre et l'auteur du document

% L'argument optionnel (entre crochets) donne le titre qui sera mis sur chaque slide
\title[Meta-learning for model selection]{The principle of meta-learning for model selection in time series forecasting}
\author{Zekhov Matvei\\
\and Boris Demeshev}
% Votre nom
% L'épreuve (car on n'a pas le droit de signaler sa provenance à un concours) (là encore, l'argument optionnel apparaît sur chaque slide)
\institute[HSE]{Higher School of Economics}
\date{Session 2019} 

% On démarre le document proprement dit
\begin{document}

% La page de titre et la table des matières
% Rien d'autre à faire qu'afficher le titre
\begin{frame}
\titlepage 
\end{frame}

\begin{frame}
\frametitle{Model selection algorythm}

\begin{center}
\includegraphics[width = 0.5\linewidth]{slides/graph.PNG}
\end{center}
\end{frame}

\begin{frame}
\frametitle{Sample}

\begin{table}[]
\resizebox{\textwidth}{!}{%
\begin{tabular}{|
>{\columncolor[HTML]{91FF91}}l |l|l|l|}
\hline
\textbf{Class} & \cellcolor[HTML]{91FF91}\textbf{Quantity} & \cellcolor[HTML]{91FF91}\textbf{Length} & \cellcolor[HTML]{91FF91}\textbf{Seasonality} \\ \hline
Yearly & 2000 & 30 & 1 \\ \hline
Quarterly & 2000 & 60 & 4 \\ \hline
Monthly & 2000 & 156 & 12 \\ \hline
Weekly & 286 & 315 & 4 \\ \hline
Daily & 2000 & 500 & 7 \\ \hline
Hourly & 245 & 720 & 24 \\ \hline
Total8531 &  &  &  \\ \hline
\end{tabular}%
}
\end{table}
\end{frame}

\begin{frame}
\frametitle{Time series exapmples}

\begin{center}
\includegraphics[width = \linewidth]{slides/time_series.pdf}
\end{center}
\end{frame}

\begin{frame}
\frametitle{Correlation of features}

\begin{center}
\includegraphics[width = 0.8\linewidth]{slides/corr.png}
\end{center}
\end{frame}

\begin{frame}
\frametitle{UMAP embeddings}

\begin{center}
\includegraphics[width = \linewidth]{slides/umap.pdf}
\end{center}
\end{frame}


\begin{frame}
\frametitle{PCA}

\begin{center}
\includegraphics[width = \linewidth]{slides/pca.pdf}
\end{center}
\end{frame}


\begin{frame}
\frametitle{Strength of features}

\begin{center}
\includegraphics[width = 0.8\linewidth]{slides/fin_str.pdf}
\end{center}
\end{frame}




\begin{frame}
\frametitle{New series generation}

\begin{center}
\includegraphics[width = 0.8\linewidth]{slides/generation_grid.pdf}
\end{center}
\end{frame}

\begin{frame}
\frametitle{Generated example}

\begin{center}
\includegraphics[width = 0.8\linewidth]{slides/gen_and_neighbours.pdf}
\end{center}
\end{frame}

\begin{frame}
\frametitle{Optimization process}

\begin{center}
\includegraphics[width = 0.8\linewidth]{slides/optim.pdf}
\end{center}
\end{frame}

\begin{frame}
\frametitle{All generated series}

\begin{center}
\includegraphics[width = 0.5\linewidth]{slides/all_gen.pdf}
\end{center}
\end{frame}


\begin{frame}
\frametitle{SMAPE of each model}

\begin{figure}
	\animategraphics[autoplay,controls,loop,scale=1,width=0.8\textwidth]{1}{slides/image}{1}{9}
\end{figure}
\end{frame}



\begin{frame}
\frametitle{Most acceptable model}
\begin{table}[]
\resizebox{\textwidth}{!}{%
\begin{tabular}{|
>{\columncolor[HTML]{91FF91}}l |
>{\columncolor[HTML]{FFFFFF}}l |
>{\columncolor[HTML]{FFFFFF}}l |
>{\columncolor[HTML]{FFFFFF}}l |
>{\columncolor[HTML]{FFFFFF}}l |
>{\columncolor[HTML]{FFFFFF}}l |}
\hline
\textbf{Model\textbackslash{}Class} & \cellcolor[HTML]{91FF91}\textbf{Yearly} & \cellcolor[HTML]{91FF91}\textbf{Quarterly} & \cellcolor[HTML]{91FF91}\textbf{Monthly} & \cellcolor[HTML]{91FF91}\textbf{Weekly} & \cellcolor[HTML]{91FF91}\textbf{Daily} \\ \hline
\textbf{Naive} & 204 | 9\% & 161 | 7\% & 112 | 5\% & 39 | 10\% & 90   | 4\% \\ \hline
\textbf{Seasonal Naive} & -- & 232 |10\% & 266 | 12\% & 39 | 10\% & 230 | 10\% \\ \hline
\textbf{RW with drift} & 333 | 14\% & 339 | 15\% & 271 | 12\% & 75 | 19\% & 595 | 26\% \\ \hline
\textbf{SES} & 112 | 5\% & 89   | 3\% & 86   | 4\% & 19 | 5\% & 88   | 4\% \\ \hline
\textbf{ETS} & 311 | 14\% & 288 | 13\% & 328 | 14\% & 24 | 6\% & 56   | 2\% \\ \hline
\textbf{ARIMA} & 366 | 16\% & \cellcolor[HTML]{91FF91}443 | 19\% & \cellcolor[HTML]{91FF91}466 | 20\% & 57 | 15\% & 247 | 12\% \\ \hline
\textbf{Theta} & 166 | 7\% & 127 | 6\% & 100 | 4\% & 11 | 3\% & 152 | 7\% \\ \hline
\textbf{TBATS} & \cellcolor[HTML]{91FF91}444 | 19\% & 288 | 13\% & 323 | 14\% & 40 | 10\% & 181 | 8\% \\ \hline
\textbf{Neural net} & 348 | 15\% & 127 | 6\% & 345 | 15\% & \cellcolor[HTML]{91FF91}82 | 21\% & \cellcolor[HTML]{91FF91}655 | 29\% \\ \hline
\textbf{Total} & 2284 & 2297 & 2297 & 386 & 2294 \\ \hline
\end{tabular}%
}
\end{table}
\end{frame}



\begin{frame}
\frametitle{Minimal SMAPE}
\begin{table}[]
\resizebox{\textwidth}{!}{%
\begin{tabular}{|l|l|l|l|l|l|l|l|l|l|l|}
\hline
\rowcolor[HTML]{91FF91} 
\textbf{Model\textbackslash{}Class} & \multicolumn{2}{l|}{\cellcolor[HTML]{91FF91}\textbf{Yearly}} & \multicolumn{2}{l|}{\cellcolor[HTML]{91FF91}\textbf{Quarterly}} & \multicolumn{2}{l|}{\cellcolor[HTML]{91FF91}\textbf{Monthly}} & \multicolumn{2}{l|}{\cellcolor[HTML]{91FF91}\textbf{Weekly}} & \multicolumn{2}{l|}{\cellcolor[HTML]{91FF91}\textbf{Daily}} \\ \hline
\rowcolor[HTML]{91FF91} 
\textbf{} & \textbf{Mean} & \textbf{Sd} & \textbf{Mean} & \textbf{Sd} & \textbf{Mean} & \textbf{Sd} & \textbf{Mean} & \textbf{Sd} & \textbf{Mean} & \textbf{Sd} \\ \hline
\rowcolor[HTML]{FFFFFF} 
\cellcolor[HTML]{91FF91}\textbf{Naive} & 0.732 & 0.637 & 0.469 & 0.467 & 0.541 & 0.418 & 0.533 & 0.417 & 0.550 & 0.540 \\ \hline
\rowcolor[HTML]{FFFFFF} 
\cellcolor[HTML]{91FF91}\textbf{Seasonal Naive} & -- & -- & 0.646 & 0.482 & 0.615 & 0.452 & 0.759 & 0.541 & 0.838 & 0.566 \\ \hline
\rowcolor[HTML]{FFFFFF} 
\cellcolor[HTML]{91FF91}\textbf{RW with drift} & 0.248 & 0.334 & 0.320 & 0.301 & 0.452 & 0.420 & 0.510 & 0.394 & 0.337 & 0.310 \\ \hline
\rowcolor[HTML]{FFFFFF} 
\cellcolor[HTML]{91FF91}\textbf{SES} & 0.369 & 0.482 & 0.341 & 0.339 & 0.462 & 0.422 & 0.441 & 0.467 & 0.349 & 0.421 \\ \hline
\rowcolor[HTML]{91FF91} 
\textbf{ETS} & 0.126 & 0.173 & 0.301 & 0.326 & \cellcolor[HTML]{FFFFFF}0.410 & \cellcolor[HTML]{FFFFFF}0.399 & 0.298 & 0.382 & 0.286 & 0.314 \\ \hline
\rowcolor[HTML]{FFFFFF} 
\cellcolor[HTML]{91FF91}\textbf{ARIMA} & 0.238 & 0.374 & 0.309 & 0.326 & \cellcolor[HTML]{91FF91}0.408 & \cellcolor[HTML]{91FF91}0.383 & 0.412 & 0.372 & 0.388 & 0.420 \\ \hline
\rowcolor[HTML]{FFFFFF} 
\cellcolor[HTML]{91FF91}\textbf{Theta} & 0.406 & 0.500 & 0.639 & 0.501 & 0.729 & 0.511 & 0.730 & 0.498 & 0.784 & 0.618 \\ \hline
\rowcolor[HTML]{FFFFFF} 
\cellcolor[HTML]{91FF91}\textbf{TBATS} & 0.233 & 0.343 & 0.312 & 0.358 & 0.438 & 0.429 & 0.424 & 0.437 & 0.486 & 0.459 \\ \hline
\rowcolor[HTML]{FFFFFF} 
\cellcolor[HTML]{91FF91}\textbf{Neural Net} & 0.454 & 0.511 & 0.493 & 0.430 & 0.509 & 0.425 & 0.683 & 0.473 & 0.605 & 0.474 \\ \hline
\end{tabular}%
}
\end{table}
\end{frame}

\begin{frame}
\frametitle{Thank you!}

\begin{center}
\includegraphics[width = 0.5\linewidth]{slides/poiss.jpg}
\end{center}
\end{frame}

% La première grande partie: introduction du sujet

\end{document}


